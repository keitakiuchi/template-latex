% =======================  論文メタデータ用テンプレート  =======================
% ファイル名例: manuscript.tex
% コンパイル例: xelatex manuscript.tex   あるいは   lualatex manuscript.tex
% ---------------------------------------------------------------------------

\documentclass[12pt,a4paper]{article}

% ----------  基本設定 ----------
\usepackage{geometry}        % 余白調整
\usepackage{setspace}        % 行間
\usepackage{hyperref}        % 参照リンク
\usepackage{authblk}         % 著者・所属
\usepackage{titlesec}        % セクション間隔微調整

% ----------  多言語対応 ----------
\usepackage{polyglossia}
\setmainlanguage{english}    % 主言語
\setotherlanguage{japanese}  % 副言語
%
\usepackage{fontspec}
\defaultfontfeatures{Scale=MatchLowercase}
\setmainfont{Times New Roman}            % 欧文本文
\newfontfamily\japanesefont{Noto Serif CJK JP}  % 和文本文
\newfontfamily\japanesefontsf{Noto Sans CJK JP} % 和文サンセリフ体

% ----------  ORCID 用(任意) ----------
\usepackage{academicons}
\usepackage{xcolor}
\newcommand{\orcidicon}[1]{%
  \href{https://orcid.org/#1}{\textcolor{black}{\aiOrcid}}}

% ----------  著者・タイトル情報 ----------
\title{\bfseries [ここに論文フルタイトルを入力]}
\newcommand{\shorttitle}{[ヘッダー用の短縮タイトルを入力]} % 40 文字以内
\date{} % 日付は空欄(必要なら入力)

% --- 著者ブロック ---
% 例: \author[1]{Taro Yamada\textsuperscript{*} (he/him) \orcidicon{0000-0000-0000-0000}}
% 上付き数字は \affil の番号と対応
\author[1]{First Name Last Name\textsuperscript{*} (pronouns: he/him) \orcidicon{0000-0000-0000-0000}}
\author[2]{Second Name Last Name \orcidicon{0000-0000-0000-0000}}
\author[1,3]{Third Name Last Name}

% --- 所属ブロック ---
\affil[1]{Department of Psychology, Faculty of Health Sciences, Example University, Tokyo, Japan}
\affil[2]{Institute of Cognitive Science, ABC University, New York, USA}
\affil[3]{Graduate School of AI Studies, XYZ Institute, Kyoto, Japan}

% --- ヘッダーに短縮タイトルと著者名(任意) ---
\usepackage{fancyhdr}
\pagestyle{fancy}
\fancyhf{}  % クリア
\fancyhead[L]{\shorttitle}
\fancyhead[R]{First Name Last Name et al.}
\fancyfoot[C]{\thepage}

% ----------  体裁微調整(オプション) ----------
\setstretch{1.25}
\titlespacing*{\section}{0pt}{\baselineskip}{0.5\baselineskip}

% =======================  ドキュメント開始  =======================
\begin{document}
\maketitle

\begin{abstract}
ここに要旨(英語または日本語)を入力してください。
\end{abstract}

\section*{Data Availability Statement}
The datasets and analysis scripts are available in the OSF repository (\url{https://doi.org/xxx}) and will be made public upon publication.

\section*{Funding Statement}
This work was supported by JSPS KAKENHI Grant Number JP20H01234.

\section*{Conflict of Interest Disclosure}
The authors declare no conflicts of interest.

\section*{Ethics Approval Statement}
This study did not involve human participants, animals, or identifiable personal data and therefore did not require ethics committee review.

\section*{Permission to Reproduce Material / Clinical Trial Registration}
Not applicable.

% --------  以下は必要に応じて削除または追加 --------
\section*{Author Contribution Note (Optional)}
The first author led conceptualization and drafting; all subsequent authors are listed alphabetically and contributed equally.

\section*{Word Count / Figures / Tables (Optional)}
Word count: 4,321 Figures: 2 Tables: 1

\section*{Acknowledgements (Optional)}
The authors thank Dr.\ Hanako Suzuki for helpful comments on earlier drafts.

% =======================  参考文献など  =======================
\bibliographystyle{apa}   % お好みのスタイルに変更可
\bibliography{references} % refs.bib を別途用意

\end{document}
% =======================  テンプレートここまで  =======================
