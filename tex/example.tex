% example.tex - 汎用的なLaTeXテンプレート例
% 日本語論文・レポート作成のための基本テンプレート
% このファイルをコピーして、プロジェクトに合わせてカスタマイズしてください

\documentclass[a4paper,11pt]{article}

% 既存の共通プリセット
% preamble.tex - パッケージ設定

% LuaLaTeX 用日本語設定
\usepackage{luatexja}
\usepackage{luatexja-fontspec}

% 日本語フォント設定
\setmainjfont[BoldFont=IPAPGothic]{IPAMincho}
\setsansjfont{IPAPGothic}

% 数式
\usepackage{amsmath}
\usepackage{amssymb}
\usepackage{mathtools}

% 図表
\usepackage{graphicx}
\graphicspath{{../figures/}}
\usepackage{float}
\usepackage{subcaption}
\usepackage{booktabs}

% 参考文献
\usepackage{url}
\usepackage{hyperref}

% その他の便利なパッケージ
\usepackage{enumitem}
\usepackage{listings}
\usepackage{xcolor}

% コードのハイライト設定
\definecolor{codegreen}{rgb}{0,0.6,0}
\definecolor{codegray}{rgb}{0.5,0.5,0.5}
\definecolor{codepurple}{rgb}{0.58,0,0.82}
\definecolor{backcolour}{rgb}{0.95,0.95,0.92}

\lstdefinestyle{mystyle}{
    backgroundcolor=\color{backcolour},
    commentstyle=\color{codegreen},
    keywordstyle=\color{codepurple},
    numberstyle=\tiny\color{codegray},
    stringstyle=\color{codegreen},
    basicstyle=\ttfamily\footnotesize,
    breakatwhitespace=false,
    breaklines=true,
    captionpos=b,
    keepspaces=true,
    numbers=left,
    numbersep=5pt,
    showspaces=false,
    showstringspaces=false,
    showtabs=false,
    tabsize=2
}

\lstset{style=mystyle}

% ハイパーリンクの設定
\hypersetup{
    colorlinks=true,
    linkcolor=blue,
    filecolor=magenta,
    urlcolor=cyan,
}

% ページ設定
\usepackage{geometry}
\geometry{
    a4paper,
    left=30mm,
    right=30mm,
    top=30mm,
    bottom=30mm,
} 

% ---------- 言語・日本語設定 ----------
\usepackage{luatexja}           % 日本語
\usepackage{luatexja-fontspec}
\usepackage{fontspec}           % OTF/TTF フォント
\usepackage{polyglossia}        % 言語自動切替
\setmainlanguage{english}
\setotherlanguage{japanese}

% 日本語フォントの設定を簡素化
\defaultjfontfeatures{JFM=ujis}

% ---------- 数式・図・リンク ----------
\usepackage{amsmath, amssymb}
\usepackage{graphicx}
\usepackage{hyperref}

% ---------- 参考文献(BibTeX) ----------
\usepackage{cite}        % 引用コマンド
\bibliographystyle{plain} % 参考文献スタイル

% ---------- 本文開始 ----------
\begin{document}

\title{論文タイトル / Paper Title}
\author{著者名 / Author Name}
\date{\today}
\maketitle

\tableofcontents
\clearpage

% 各章を読み込む
\input{sections/intro}

% 引用例(テスト用)
\section{引用例}
ここに引用を入れてみましょう \cite{example2024}。
また、別の文献も引用できます \cite{examplebook2024}。

% ---------- テーブル例(CSV抽出テスト用) ----------
\section{テーブル例}

% テーブル1: 基本的なテーブル
\begin{table}[h]
\centering
\caption{基本的なテーブル例}
\begin{tabular}{|l|c|r|}
\hline
項目 & 数値 & 単位 \\
\hline
長さ & 10.5 & cm \\
幅 & 5.2 & cm \\
高さ & 3.1 & cm \\
\hline
\end{tabular}
\end{table}

% テーブル2: 数式を含むテーブル
\begin{table}[h]
\centering
\caption{数式を含むテーブル例}
\begin{tabular}{|l|c|c|}
\hline
計算式 & 結果 & 備考 \\
\hline
$2 + 3$ & 5 & 基本演算 \\
$10 \times 5$ & 50 & 乗算 \\
$\sqrt{16}$ & 4 & 平方根 \\
$3.14159 \pm 0.00001$ & 3.14 & 円周率 \\
\hline
\end{tabular}
\end{table}

% テーブル3: 日本語を含むテーブル
\begin{table}[h]
\centering
\caption{日本語を含むテーブル例}
\begin{tabular}{|l|c|r|}
\hline
\textbf{項目名} & \textit{値} & \emph{説明} \\
\hline
データ1 & 100 & 最初のデータ \\
データ2 & 200 & 二番目のデータ \\
データ3 & 300 & 三番目のデータ \\
\hline
\end{tabular}
\end{table}

\input{sections/conclusion}

% ---------- 参考文献一覧 ----------
\bibliography{bib/example}

\end{document}
